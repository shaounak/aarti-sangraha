% !TEX program = xelatex

\documentclass[letterpaper,twocolumn,openany,nodeprecatedcode]{dndbook}
\usepackage[T1]{fontenc}
% Use babel or polyglossia to automatically redefine macros for terms
% Armor Class, Level, etc...
% Default output is in English; captions are located in lib/dndstrings.sty.
% If no captions exist for a language, English will be used.
%1. To load a language with babel:
%	\usepackage[<lang>]{babel}
%2. To load a language with polyglossia:
\usepackage{polyglossia}
\usepackage{fontspec}

% \setmainlanguage[numerals=devanagari]{marathi}
\setmainlanguage{marathi}
% \setmainfont{Shobhika}
\setmainfont[Script=Devanagari, Mapping=devanagarinumerals]{Shobhika}
% \setmainfont{Lohit Marathi}
\setotherlanguages{english}
% Define the Latin font
% \setmainfont{Noto Serif}

% Define the Devanagari font (Noto Sans Devanagari)
% \newfontfamily\devanagarifont{Noto Sans Devanagari}[
%   Script=Devanagari,
%   Renderer=HarfBuzz % Use HarfBuzz for complex scripts like Devanagari
% ]

% \newfontfamily\marathifont{Noto Sans Devanagari}[Script=Devanagari]
% \newfontfamily\marathifont{NotoSansDevanagari-Regular}[Script=Devanagari]
% \newfontfamily\englishfont{Times New Roman}
% \defaultfontfeatures{Mapping=tex-text}
% \usepackage[english]{babel}
%\usepackage[italian]{babel}
% For further options (multilanguage documents, hypenations, language environments...)
% please refer to babel/polyglossia's documentation.

\usepackage[utf8]{inputenc}
\usepackage[singlelinecheck=false]{caption}
\usepackage{lipsum}
\usepackage{listings}
\usepackage{shortvrb}
\usepackage{stfloats}

\captionsetup[table]{labelformat=empty,font={sf,sc,bf,},skip=0pt}

\MakeShortVerb{|}

\lstset{%
  basicstyle=\ttfamily,
  language=[LaTeX]{TeX},
  breaklines=true,
}

% \title{ 
%   \devanagarifont ।। मराठी आरती संग्रह ।। \\
% }
% \date{}
\title{ मराठी आरती संग्रह } \author{ शौनक नाशिककर } \date{\today}

\raggedbottom
\begin{document}

\frontmatter

\maketitle

\tableofcontents

\mainmatter%

\part{आरती}

% \chapter{आरती}
\chapter{गणपतीची आरती(Ganpati Aarti)}
% \begin{multicols}{2}
\subsection*{marathi}
\begin{flushleft}
  % \rule{\linewidth}{1pt} \\
  सुखकर्ता दुखहर्ता, वार्ता विघ्नांची । \\
  नुरवी; पुरवी प्रेम, कृपा जयाची ।\\
  सर्वांगी सुंदर, उटी शेंदुराची । \\
  कंठी झळके माळ, मुक्ताफळांची ॥ १ ॥ \\
  % -------------------------------------- \\
  \rule{\linewidth}{1pt} \\
  जय देव, जय देव जय मंगलमूर्ती, हो श्री मंगलमूर्ती । \\
  दर्शनमात्रे मनकामना पुरती ॥धृ॥ \\
  \rule{\linewidth}{1pt} \\
  % -------------------------------------- \\
  रत्नखचित फरा तूज गौरीकुमरा । \\
  चंदनाची उटी कुंकुमकेशरा । \\
  हिरे जडित मुकुट शोभतो बरा । \\
  रुणझुणती नुपुरे चरणी घागरिया । \\
  चरणीच्या घागरिया रुणझुण वाजती । \\
  तेथे नांदे देवा अंबर गर्जती । \\
  ताता ठुमकत ठुमकत नाचे गणपती । \\
  तव शंकर पार्वती कौतुक पाहाती ॥ २ ॥ \\
  \rule{\linewidth}{1pt} \\
  लंबोदर पीतांबर, फणिवरबंधना । \\
  सरळ सोंड, वक्रतुंड त्रिनयना । \\
  दास रामाचा, वाट पाहे सदना । \\
  संकटी पावावे, निर्वाणी रक्षावे, सुरवरवंदना । \\
  जय देव जय देव, जय मंगलमूर्ती । \\
  दर्शनमात्रे मनकामना पुरती ॥ ३ ॥
  \pagebreak
  \subsection*{english}
  sukhakarta dukhaharta, varta vighnanchi । \\
  nurvi; purvi prem, kripa jayachi ।।\\
  sarvangi sundar, uti shendurachi ।\\
  kanthi zalke mal, muktaphalanchi ।। 1 ।। \\
  \rule{\linewidth}{1pt} \\
  jai dev, jay dev jay mangalamoorthy, ho shri mangalmurti . \\ darshanamatre man kamna purati
  \rule{\linewidth}{1pt} \\
  ratnakhachit fara tuj gaureekumara । \\
  chandanachi uti kunkumakeshara ।। \\
  hire jadit mukut shobhato baraa । \\
  runajhunati nupure charani ghagariya । \\
  charanichya ghagariya runajhun vajtie ।। \\
  tethe nande deva amber garjati । \\
  tata thumakat thumakat nache ganpati ।। \\
  tav shankar parvati kautuk pahatis ।। 2 ।।\\
  \rule{\linewidth}{1pt} \\
  ambodar pitambar, phanivarabandhana । \\
  saral sond, vakratund trinyanaa ।। \\
  das ramacha, vat pahe sadanaa । \\
  sankati pavave, nirvani rakshave, suravaravandana ।। \\
  jai dev jai dev, jay mangalamoorthy । \\
  darshanamatre manakamana purti ।। 2 ।। \\
\end{flushleft}
% \end{multicols}
% -------------------------------------- \\
\pagebreak
\chapter{दुर्गा देवीची आरती(Durga Devi Aarti)}
\begin{flushleft}
  \subsection*{marathi}
  % \rule{\linewidth}{1pt} \\
  % \chapter{दुर्गा देवीची आरती
  दुर्गे दुर्घट भारी तुजविण संसारी । \\
  अनाथनाथे अंबे करुणा विस्तारी ॥\\
  वारी वारीं जन्ममरणाते वारी । \\
  हारी पडलो आता संकट नीवारी ॥ १ ॥ \\
  \rule{\linewidth}{1pt} \\
  जय देवी जय देवी जय महिषासुरमथनी, \\ ओ दैत्यासुरमथनी । \\
  सुरवरईश्वरवरदे तारक संजीवनी ॥धृ॥ \\
  \rule{\linewidth}{1pt} \\
  त्रिभुवनी भुवनी पाहतां तुज ऎसे नाही । \\
  चारी श्रमले परंतु न बोलावे काहीं ॥ \\
  साही विवाद करितां पडिले प्रवाही । \\
  ते तूं भक्तालागी, ते तूं दासालागी, पावसि लवलाही ॥२॥ \\
  \rule{\linewidth}{1pt} \\
  प्रसन्न वदने प्रसन्न होसी निजदासां । \\
  क्लेशापासूनि सोडी, दुःखांपासुन सोडवी, तोडी भवपाशा ॥ \\
  अंवे तुजवांचून कोण पुरविल आशा । \\
  नरहरि तल्लिन झाला पदपंकजलेशा ॥३॥ \\
  \pagebreak
  \subsection*{english}
  durge durghat bhari tujavin sansari \\ anathanathe ambe karuna vistaris\\ vari varin janmamaranate varie \\ hari padalo ata sankat nivari \\
  \rule{\linewidth}{1pt} \\
  jay devi jay devi jay mahishasuramathani, \\ o daityasuramathani \\ suravaraishvaravarade tarak sanjivani \\
  \rule{\linewidth}{1pt} \\
  tribhuvani bhuvani pahatan tuj se nahi \\ chari shramale parantu n bolave kahin \\ sahi vivad karitan padile pravahi \\ te toon bhaktalagi, te toon dasalagi, pavasi lavalahi \\
  \rule{\linewidth}{1pt} \\
  prasann vadane prasann hosie nijadasan \\ kleshapasuni sodi, duhkhaanpaasun sodavi, todi bhavpasha \\ anve tujavanchun kon purvil ashaa \\ narhari tallin jhala padapankajaleshaan \\

\end{flushleft}
\pagebreak
\chapter{शंकराची आरती(Shankar Aarti)}
\begin{flushleft}
  \subsection*{marathi}
  लवथवती विक्राळा ब्रह्मांडी माळा। \\
  वीषें कंठी काळा त्रिनेत्रीं ज्वाळां॥ \\
  लावण्यसुंदर मस्तकीं बाळा। \\
  तेथुनियां जळ निर्मळ वाहे झुळझुळां॥१॥ \\
  \rule{\linewidth}{1pt} \\
  जय देव जय देव जय श्रीशंकरा, ओ स्वामी शंकरा।। \\
  आरती ओवाळूं तुज कर्पुगौरा ।।ध्रु.।। \\
  \rule{\linewidth}{1pt} \\
  कर्पुगौरा भोळा नयनीं विशाळा। \\
  आर्धांगीं पार्वती सुमनांच्या माळा। \\
  विभुतीचें उधळण शितिकंठ नीळा। \\
  ऐसा शंकर शोभे उमावेल्हाळा॥ जय.॥२॥ \\
  \rule{\linewidth}{1pt} \\
  देवी दैत्यीं सागरमंथन पैं केलें। \\
  त्यामाजी जें अवचित हळाहळ उठिलें। \\
  तें त्वां असुरपषं प्राशन केलें। \\
  नीळकंठ नाम प्रसिद्ध झालें॥जय.॥३॥ \\
  \rule{\linewidth}{1pt} \\
  व्याघ्रांबर फणिवरधर सुंदर मदनारी।\\
  पंचानन मनमोहन मुनिजनसुखकारी।\\
  शतकोटीचें बीज वाचे उच्चारी।\\
  रघुकुलतिलक रामदासा अंतरी॥\\
  जय देव जय देव जय श्रीशंकरा॥४॥
  \pagebreak
  \subsection*{english}
  lavathavati vikrala brahmandi mala \\ vishen kanthi kala trinetrin jvalan \\ lavanyasundar mastakin bala \\ tethuniyan jal nirmal vahe jhulajhulan \\
  \rule{\linewidth}{1pt} \\
  jay dev jay dev jay shreeshankara, o swami shankara \\ aarati ovalu tuj karpugaura iadhru.. \\
  \rule{\linewidth}{1pt} \\
  karpugaura bhola nayanin vishala \\ ardhangin parvati sumanaanchya mala \\ vibhutichen udhalan shitikanth nilaa \\ aisa shankar shobhe umavelhala jay \\
  \rule{\linewidth}{1pt} \\
  devi daityin sagaramanthan pain kelen \\ tyamaji jen avachit halahal uthilen \\ ten tvan asurapashan prashan kelen \\ nilakanth naam prasiddh jhalenka \\
  \rule{\linewidth}{1pt} \\
  vyaghrambar phanivaradhar sundar madanari\\ panchanan manmohan munijansukhkari\\ shatakotichen bij vache uchchari\\ raghukulatilak ramadasa antaree\\ jay dev jay dev jay shreeshankara

\end{flushleft}
\chapter{महालक्ष्मीची आरती (Mahalaxmi Aarti )}
\begin{flushleft}
  \subsection*{marathi}
  जय देवी जय देवी जय महालक्ष्मी \\
  वससी व्यापकरूपे राहे निश्चलरूपे तू स्थूलसुक्ष्मी. जय. \\
  \rule{\linewidth}{1pt} \\
  करवीरपूर वासिनी सुरवर मुनिमाता \\
  पुरहर वरदायिनी मुरहर प्रियकांता \\
  कमलाकारे जठरी जन्मविला धाता \\
  सहस्त्र वदनी भूधर नपुरे गुणगाता.॥1॥ जय. \\
  \rule{\linewidth}{1pt} \\
  मातुल्लिंग गदा खेटक रविकिरणी \\
  झळके हाटकवाटी पीयुष रसपाणि \\
  माणिक रसना सुरंग वसना मृगनयनी \\
  शशीकर वदना राजस मदनाची जननी.॥२॥ जय. \\
  \rule{\linewidth}{1pt} \\
  तारा शक्ती अगम्या शीवभजका गौरी \\
  सांख्य म्हणती प्रकृती निर्गुण निर्धारी \\
  गायत्री नीजबीजा निगमागमसारी \\
  प्रगटे पद्मावती निजधर्माचारी. जय.॥3॥ \\
  \rule{\linewidth}{1pt} \\
  अमृत भरिते सरिते अघदुरिते वारी \\
  मारी दुर्घट असुरा भवदुस्तर तारी \\
  वारी माया पटल प्रणमत परिवारी \\
  हे रूप चिद्रुप दावी निर्धारी. जय.॥4॥ \\
  \rule{\linewidth}{1pt} \\
  चतुराननाने कुश्चित कर्मांच्या ओळी \\
  लिहिल्या असतिल माते माझे निजभाळी \\
  पुसोनि चरणातळी पदसुमने क्षाळी \\
  मुक्तेश्वर नागर क्षीरसागर बाळी. जय.॥5॥ \\
  \pagebreak
  \subsection*{english}
  jay devi jay devi jay mahalakshmi \\ vasasi vyapakarupe rahe nishchalarupe tu sthulasukshmi.॥1॥ jay. \\
  \rule{\linewidth}{1pt} \\
  karvirpur vasini suravar munimata \\ purhar vardayini murahar priyakanta \\ kamalakare jathari janmavila dhata \\ sahastra vadani bhudhar napure gunagata.॥२॥ jay. \\
  \rule{\linewidth}{1pt} \\
  matulling gada khetak ravikirani \\ jhalake hatakavati piyush raspani \\ manik rasna surang vasana mrignayani \\ shasheekar vadana rajas madanaachi janani.॥3॥ jay. \\
  \rule{\linewidth}{1pt} \\
  tara shakti agamya shivabhajaka gauri \\ sankhya mhanati prakrti nirgun nirdhari \\ gayatri nijabija nigamagamasari \\ pragate padmavati nijadharmachari.॥4॥ jay. \\
  \rule{\linewidth}{1pt} \\
  amrit bharite sarite aghdurite vari \\ mari durghat asura bhavdustar tari \\ vari maya patal pranamat parivari \\ he roop chidrup davi nirdhari.॥5॥ jay. \\
  \rule{\linewidth}{1pt} \\
  chaturannane kushchit karmanchya oli \\ lihilya asatil mate majhe nijabhali \\ pusoni charanaatali padsumne ksali \\ mukteshwar nagar kshirsagar bali.॥6॥ jay. \\
\end{flushleft}
% \end{start}\
\pagebreak
\chapter{श्री गुरुदत्ताची आरती(Shri Gurudattaachi Aarti)}
\begin{flushleft}
  \subsection*{marathi}
  त्रिगुणात्मक त्रैमूर्ती दत्त हा जाणा।  \\
  त्रिगुणी अवतार त्रेलोक्य राणा।। \\
  नेती नेती शब्द न ये अनुमाना।  \\
  सुरवर मुनिजन योगी समाधी न ये ध्याना।।१।। \\
  \rule{\linewidth}{1pt} \\
  जय देव जयदेव जय श्री गुरुदत्ता।\\ आरती ओवाळिता हरली भवचिंता। जय देव जय देव।
  \rule{\linewidth}{1pt} \\
  सबाह्य अभ्यंतरीं तू एक दत्त।\\ अभाग्यासी कैंची कळेल ही मात।।\\
  परा ही परतली तेथे कैंचा हा हेत।\\ जन्मरमरण्याचा पुरलासे अंत ।।२।। जय देव जय देव
  \rule{\linewidth}{1pt} \\
  दत्त येऊनिया उभा ठाकला।\\ सद्भावे साष्टांग प्रणिपात केला।।\\
  प्रसन्न होऊनिया आशीर्वाद दिधला।\\ जन्ममरण्याचा फेरा चुकविला।।३।। जय देव जय देव
  \rule{\linewidth}{1pt} \\
  दत्त दत्त ऐसे लागले ध्यान।\\ हारपले मन झाले उन्मन।। \\
  मीतूंपणाची झाली बोळवण।\\ एका जनार्दनीं श्री दत्त ध्यान ।।४।। जय देव जय देव
  \pagebreak
  \subsection*{english}
  trigunatmak traimurti datta ha janaa \\ triguni avatar treloky rana \\ neti neti shabd n ye anumana \\ suravar munijan yogi samadhi n ye dhyana \\
  \rule{\linewidth}{1pt} \\
  jai dev jaydev jai shri gurudatta\\ arati ovalita harali bhavachintaa. jai dev jai dev
  \rule{\linewidth}{1pt} \\
  sabahy abhyantarin tu ek datta\\ abhagyasi kainchi kalel he mata\\ para he paratali tethe kaincha ha heta\\ janmaramaranyacha purlase ant. jai dev jai dev
  \rule{\linewidth}{1pt} \\
  datt yeuniya ubha thakala\\ sadbhave sashtang pranipat kelaa\\ prasann houniya ashirvad didhalaa\\ janmamaranyacha ferra chukvila. jai dev jai dev
  \rule{\linewidth}{1pt} \\
  datt datt aise lagale dhyana\\ harpale man jhale unmana \\ mitunpanachi jhali bolavana\\ eka janardanin shri datt dhyan. jai dev jai dev
\end{flushleft}
\pagebreak
\chapter{विठ्ठलाची आरती(Vitthalachi Aarti)}
\begin{flushleft}
  \subsection*{marathi}
  येई हो विठ्ठले माझे माउली ये।।\\
  निढळावरी कर, ठेवूनि वाट मी पाहे।।\\
  आलिया गोलिया हाती धाडी निरोप।\\
  पंढरपुरी आहे, माझा मायबाप।।१।।\\
  \rule{\linewidth}{1pt} \\
  पिवळा पितांबर कैसा गगनी झळकला।\\
  गरुडावरी बैसोनी, माझा कैवारी आला।।२।।\\
  \rule{\linewidth}{1pt} \\
  विठोबाचे राज्य आम्हा नित्य दिपवाळी।। \\
  विष्णुदास नामा, जीवेभावे ओवाळी।।३।। \\
  \rule{\linewidth}{1pt} \\
  असो नासो भाव आम्हा तुज्या थाया\\
  कृपाद्रिष्टि पाहे, माझा पंढरीराया\\
  येई हो विठ्ठले माझे माऊली ये॥\\
  \pagebreak
  \subsection*{english}
  yei ho viththale majhe mauli yeah\\ nidhalavari kar, thevuni vat mi pahei\\ alia goliya hati dhadi niropa\\ pandharpuri aahe, mazza maybapa।।1।।\\
  \rule{\linewidth}{1pt} \\
  pivala pitambar kaisa gagani jhalakalaa\\ garudavari basoni, mazza kaivari alaa।।2।।\\
  \rule{\linewidth}{1pt} \\
  vithobache rajy amha nitya dipavali।।\\ vishnudas nama jivebhave ovali।।3।।
  \rule{\linewidth}{1pt} \\
  aso naso bhav amha tujya thaya\\ kripadristi pahe, mazza pandhariraya\\ yei ho viththale majhe mauli ye\\
\end{flushleft}
\pagebreak
\chapter{कापूर आरती(Kapoor Aarti)}
\begin{flushleft}
  \subsection*{marathi}
  कर्पूरगौरं करुणावतारं।\\
  संसारसारम् भुजगेन्द्रहारम्॥ \\
  सदावसन्तं हृदयारविन्दे।\\
  भवं भवानीसहितं नमामि॥\\
  मंदारमाला कुलतालकायै: कपालमालांकृत शेखरायः \\
  दिव्याम्बराये च दिगम्बराय, नमः शिवाय च नमः शिवाय॥ \\
  \pagebreak
  \subsection*{english}
  \textenglish{
    karpūragauraṁ karuṇāvatāraṁ। \\
    sansārsāram bhujagendrahāram॥ \\
    sadāvasantaṁ hṛdayāravinde। \\
    bhavaṁ bhavānīsahitaṁ namāmi॥ \\
    mandaramala kultalkayai: kapalamalankrit shekharayah \\ divyambaraye ch digambaray, namah shivay ch namah shivaya \\
  }
\end{flushleft}
\pagebreak
% \part{मंत्र, श्लोक, भजन, व अथर्वशीर्ष }
\part{मंत्र, श्लोक, व भजन}
\chapter{मंत्रपुष्पांजली मंत्र (Mantra Pushpanjali )}
\begin{flushleft}
  \section*{marathi}
  ॐ यज्ञेन यज्ञमयजन्त देवास्तानि धर्माणि प्रथमान्यासन् ।\\ ते ह नाकं महिमानः सचन्त यत्र पूर्वे साध्याः सन्ति देवाः ।।1।।\\
  \rule{\linewidth}{1pt} \\
  ॐ राजाधिराजाय प्रसह्यसाहिने नमो वयं वैश्रवणाय कुर्महे ।\\ स मे कामान्कामकामाय मह्यम् कामेश्वरो वैश्रवणो ददातु ।\\ कुबेराय वैश्रवणाय महाराजाय नमः ।।2।।\\
  \rule{\linewidth}{1pt} \\
  ॐ स्वस्ति। साम्राज्यं भौज्यं स्वाराज्यं वैराज्यं पारमेष्ठ्यं राज्यं माहाराज्यमाधिपत्यमयं समंतपर्यायी स्यात्सार्वभौमः सार्वायुष आंतादापरार्धात्पृथिव्यै समुद्रपर्यंताया एकराळिति ।।3।।\\
  \rule{\linewidth}{1pt} \\
  तदप्येषः श्लोको ऽभिगीतो ।मरुतः परिवेष्टारो मरुत्तस्यावसन् गृहे । आविक्षितस्य कामप्रेर्विश्वे देवाः सभासद इति ।।4।।\\
  \rule{\linewidth}{1pt} \\
  ॐ एकदंतायविघ्महे वक्रतुण्डाय धीमहि।\\
  तन्नोदंती प्रचोदयात्।\\
  \rule{\linewidth}{1pt} \\
  ॐ तत्पुरुषाय विद्महे महादेवाय धीमहि।
  तन्नो रुद्रः प्रचोदयात्॥ \\
  \rule{\linewidth}{1pt} \\
  मंत्रपुष्पांजली समर्पयामि ।।\\

  ।। गणपतिबाप्पा मोरया ।।\\
  \pagebreak
  \section*{english}
  oṃ yajñena yajñamayajanta devāstāni dharmāṇi prathamānyāsan। \\
  te ha nākam mahimānaḥ sacanta yatra pūrve sādhyāḥ santi devāḥ।।1।।\\
  om rājādhirājāya prasahyasāhine namovayam vaiśravaṇāya kurmahe।\\
  sa me kāmānkāmakāmāya mahyam kāmeśvaro vaiśravaṇo dadātu।\\
  kuberāya vaiśravaṇāya mahārājāya namaḥ।।2।।\\
  om svasti।\\
  sāmrājyam bhaujyam svārājyam vairājyam pārameṣṭhyam rājyam
  māhārājyamādhipatyamayam samantaparyāyī syātsārvabhaumaḥ sārvāyuṣa āntādāparārdhātpṛthivyai samudraparyantāyā ekarāḷiti ।। 3 ।।
  tadapyeṣa śloko 'bhigīto ।
  marutaḥ pariveṣṭāro maruttasyāvasan gṛhe ।
  āvikśitasya kāmaprerviśve devāḥ sabhāsada iti ।। 4 ।।
  \rule{\linewidth}{1pt} \\
  oṃ ekadantayavighmahe vakratunday dhimhi\\ tannodanti prachodayat ।।\\
  % viśvataścakṣur uta viśvatomukho viśvatobāhur uta viśvataspāt । sam bāhubhyāṁ dhamati sam patatrair dyāvābhūmī janayan deva ekaḥ ।। 5 ।।
  \rule{\linewidth}{1pt} \\
  oṃ tatpuruṣāya vidmahe mahādevāya dhīmahi।\\
  tanno rudraḥ pracodayāt।।
  \pagebreak
  \chapter{घालीन लोटांगण(Ghalin Lotangan)}

  \section*{marathi}
  घालीन लोटांगण, वंदीन चरण । \\
  डोळ्यांनी पाहीन रूप तुझे । \\
  प्रेमे आलिंगिन, आनंदे पूजिन । \\
  भावें ओवाळीन म्हणे नामा ॥१॥ \\
  \rule{\linewidth}{1pt} \\
  त्वमेव माता पिता त्वमेव । \\
  त्वमेव बंधुश्च सखा त्वमेव । \\
  त्वमेव विद्या द्रविणं त्वमेव । \\
  त्वमेव सर्वं मम देवदेव ॥२॥ \\
  \rule{\linewidth}{1pt} \\
  कायेन वाचा मनसेंद्रियैर्वा । \\
  बुध्यात्मना वा प्रकृतिस्वभावात् । \\
  करोमि यद्यत्सकलं परस्मै । \\
  नारायणायेति समर्पयामि ॥३॥ \\
  \rule{\linewidth}{1pt} \\
  अच्युतं केशवं रामनारायणं । \\
  कृष्ण दामोदरं वासुदेवं हरिम् । \\
  श्रीधरं माधवं गोपिकावल्लभं । \\
  जानकीनायकं रामचंद्र भजे ॥ ४ ॥ \\
  \rule{\linewidth}{1pt} \\
  हरे राम हरे राम, राम राम हरे हरे । \\
  हरे कृष्ण हरे कृष्ण, कृष्ण कृष्ण हरे हरे ॥५॥ \\
  \pagebreak
  \section*{english}
  Ghaalin Lotaangan Vandin \\
  Ghaalin Lotaangan Vandin Charan. \\
  Dolyaanni Paahin Roop Tujhe \\
  Preme Aalingan Aanande Poojin. \\
  Bhaave Ovaalin Mhane Naama. ॥1॥\\
  \rule{\linewidth}{1pt} \\
  Tvamev Mata cha Pita Tvamev \\
  Tvamev Bandhu cha Sakha Tvamev. \\
  Tvamev Vidya Dravinam Tvamev \\
  Tvamev Sarvam Mam Dev dev. ॥2॥\\
  \rule{\linewidth}{1pt} \\
  Kaayen Vaacha Manasen-driyairva, \\
  Budhdaatmana va Prakruti-svabhavaat \\
  Karomi Yadyat Sakalam Parasmai \\
  Narayanayeti Samarpayaami. ॥3॥\\
  \rule{\linewidth}{1pt} \\
  Achayutam Keshavam Ram-narayanam, \\
  Krishna-damodaram Vasudevam Hari. \\
  Shridharam Madhavam Gopika-vallabham, \\
  Janaki-nayakam Ram-chandram Bhaje ॥4॥ \\
  \rule{\linewidth}{1pt} \\
  Hare Ram Hare Ram \\
  Ram Ram Hare Hare \\
  Hare Krishna Hare Krishna \\
  Krishna Krishna Hare Hare. ॥5॥\\
  \pagebreak
%   \chapter{Atharva Shirsh}
%   \section*{marathi}
%   ॐ भद्रं कर्णेभिः शृणुयाम देवाः ।
%   भद्रं पश्येमाक्षभिर्यजत्राः ।
%   स्थिरैरङ्गैस्तुष्टुवाग्‍ँसस्तनूभिः ।
%   व्यशेम देवहितं यदायूः ।
%   \rule{\linewidth}{1pt} \\
%   स्वस्ति न इन्द्रो वृद्धश्रवाः ।
%   स्वस्ति नः पूषा विश्ववेदाः ।
%   स्वस्ति नस्तार्क्ष्यो अरिष्टनेमिः ।
%   स्वस्ति नो बृहस्पतिर्दधातु ॥
%   ॐ शान्तिः शान्तिः शान्तिः ॥
%   ॐ नमस्ते गणपतये ॥१॥\\
%   \rule{\linewidth}{1pt} \\
%   त्वमेव प्रत्यक्षं तत्त्वमसि । \\
%   त्वमेव केवलं कर्ताऽसि । \\
%   त्वमेव केवलं धर्ताऽसि । \\
%   त्वमेव केवलं हर्ताऽसि । \\
%   त्वमेव सर्वं खल्विदं ब्रह्मासि । \\
%   त्वं साक्षादात्माऽसि नित्यम् ॥२॥ \\
%   \rule{\linewidth}{1pt} \\
%   ऋतं वच्मि । सत्यं वच्मि ॥३॥
%   \rule{\linewidth}{1pt} \\
%   अव त्वं माम् ।
%   अव वक्तारम् ।
%   अव श्रोतारम् ।
%   अव दातारम् ।
%   अव धातारम् ।
%   अवानूचानमव शिष्यम् ।
%   \rule{\linewidth}{1pt} \\
%   अव पुरस्तात् ।
%   अव दक्षिणात्तात् ।
%   अव पश्चात्तात् ।
%   अवोत्तरात्तात् ।
%   अव चोर्ध्वात्तात् ।
%   अवाधरात्तात् ।
%   सर्वतो मां पाहि पाहि समन्तात् ॥४॥
%   \rule{\linewidth}{1pt} \\
%   त्वं वाङ्मयस्त्वं चिन्मयः ।
%   त्वमानन्दमयस्त्वं ब्रह्ममयः ।
%   त्वं सच्चिदानन्दाऽद्वितीयोऽसि ।
%   त्वं प्रत्यक्षं ब्रह्मासि ।
%   त्वं ज्ञानमयो विज्ञानमयोऽसि ॥५॥
%   \rule{\linewidth}{1pt} \\
%   त्वं भूमिरापोऽनलोऽनिलो नभः ।
%   त्वं चत्वारि वाक् {परिमिता} पदानि ।
%   \rule{\linewidth}{1pt} \\
%   त्वं भूमिरापोऽनलोऽनिलो नभः ।
% त्वं चत्वारि वाक् {परिमिता} पदानि ।
% \pagebreak
% \section*{english}
%   \rule{\linewidth}{1pt} \\
%   Om Bhadram Karnnebhih Shrnnuyaama Devaah । \\
%   Bhadram Pashyema-Akssabhir-Yajatraah । \\
%   Sthirair-Anggais-Tussttuvaamsas-Tanuubhih । \\
%   Vyashema Devahitam Yad-Aayuh । \\
%   \rule{\linewidth}{1pt} \\
%   Om Namas-Te Gannapataye ।।1।।
%   Svasti Na Indro Vrddha-Shravaah । \\
%   Svasti Nah Puussaa Vishva-Vedaah । \\
%   Svasti Nas-Taarkssyo Arisstta-Nemih । \\
%   Svasti No Brhaspatir-Dadhaatu ।। \\
%   Om Shaantih Shaantih Shaantih ।। \\
%   \rule{\linewidth}{1pt} \\
%   Om Namas-Te Gannapataye ।।1।।
%   \rule{\linewidth}{1pt} \\
%   Tvam-Eva Pratyakssam Tattvam-Asi ।
%   Tvam-Eva Kevalam Kartaa-[A]si ।
%   Tvam-Eva Kevalam Dhartaa-[A]si ।
%   Tvam-Eva Kevalam Hartaa-[A]si ।
%   Tvam-Eva Sarvam Khalv[u]-Idam Brahma-Asi ।
%   Tvam Saakssaad-Aatmaa-[A]si Nityam ।।2।।
%   \rule{\linewidth}{1pt} \\
%   Svasti Na Indro Vrddha-Shravaah ।
%   Svasti Nah Puussaa Vishva-Vedaah ।
%   Svasti Nas-Taarkssyo Arisstta-Nemih ।
%   Svasti No Brhaspatir-Dadhaatu ।।
%   Om Shaantih Shaantih Shaantih ।।
%   \rule{\linewidth}{1pt} \\
%   Rtam Vacmi | Satyam Vacmi ||3||
%   Ava Tvam Maam |
%   Ava Vaktaaram |
%   Ava Shrotaaram |
%   Ava Daataaram |
%   Ava Dhaataaram |
%   Ava-Anuucaanam-Ava Shissyam |
%   \rule{\linewidth}{1pt} \\
%   Ava Purastaat |
%   Ava Dakssinnaattaat |
%   Ava Pashcaattaat |
%   Avo[a-U]ttaraattaat |
%   Ava Co[a-U]rdhvaattaat |
%   Ava-Adharaattaat |
%   Sarvato Maam Paahi Paahi Samantaat ||4||
%   \rule{\linewidth}{1pt} \\
%   Tvam Vaangmayas-Tvam Cinmayah |
%   Tvam-Aanandamayas-Tvam Brahmamayah |
%   Tvam Saccidaanandaa-[A]dvitiiyo-[A]si |
%   Tvam Pratyakssam Brahma-Asi |
%   Tvam Jnyaanamayo Vijnyaanamayo-[A]si ||5||
%   \rule{\linewidth}{1pt} \\
%   Tvam Vaangmayas-Tvam Cinmayah |
%   Tvam-Aanandamayas-Tvam Brahmamayah |
%   Tvam Saccidaanandaa-[A]dvitiiyo-[A]si |
%   Tvam Pratyakssam Brahma-Asi |
%   Tvam Jnyaanamayo Vijnyaanamayo-[A]si ||5||
%   \rule{\linewidth}{1pt} \\
%   Tvam Bhuumir-Aapo-[A]nalo-[A]nilo Nabhah |
%   Tvam Catvaari Vaak {Parimitaa} Padaani |
%   \rule{\linewidth}{1pt} \\

\end{flushleft}
\end{document}
